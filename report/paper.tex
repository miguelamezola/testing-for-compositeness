\documentclass[12pt, titlepage]{amsart}

\usepackage{amsmath, amsthm, ulem, graphicx, marvosym, fancyhdr, amscd, amssymb, enumitem, mathrsfs, multicol, setspace}
\usepackage{enumitem}
\usepackage{booktabs}
\usepackage{listings}             % Include the listings-package
\usepackage{seqsplit}
\usepackage{cleveref}


\newcommand\Z{{\mathbb Z}}
\newcommand\F{{\mathbb F}}
\newcommand\N{{\mathbb N}}
\newcommand\A{{\mathbf A}}
\newcommand\p{{\mathscr P}}
\newcommand\R{{\mathbb R}}
\newcommand\Q{{\mathbb Q}}
\newcommand\C{{\mathbb C}}
\newcommand\bfx{{\mathbf x}}
\newcommand\bfv{{\mathbf v}}
\newcommand\bfy{{\mathbf y}}
\newcommand\bfw{{\mathbf w}}

%\usepackage{graphics}

\topmargin -.7in
\evensidemargin-0in
\oddsidemargin0in
\textheight 9.5in
\textwidth 6.5in

\def\arraystretch{1.5}%  1 is the default, change whatever you need

\newtheorem{theorem}{Theorem}[subsection]
\newtheorem{lemma}{Lemma}[subsection]
\newtheorem{prop}{Proposition}[subsection]
\newtheorem{cor}{Corollary}[subsection]
\newtheorem{algorithm}{Algorithm}[subsection]
\theoremstyle{definition}
%\newtheorem{defi}{Definition}[subsection]
\newtheorem{definition}{Definition}[subsection]
\newtheorem{example}{Example}[subsection]
\newtheorem{rmk}{Remark}[subsection]
\newtheorem{interpret}{Interpretation}[subsection]
%\theoremstyle{test}
\newtheorem*{axioms}{Axioms}

%\everymath={\displaystyle}

\pagestyle{empty}
\onehalfspacing

\title{Using a Support Vector Machine to Predict Primality in Comparison with the Miller-Rabin Probabilistic Primality Test}
\author{Miguel Amezola \\
	B.S. Mathematics \\
	Pacific Lutheran University  \\
	Advisors: Dr. Tom Edgar
	%	\and
	%	Your friend who worked with you \\
	%	His/her Major / University \\
}
\email{math@miguelamezola.com}

\date{\today}

\setcounter{secnumdepth}{2}


\begin{document}
	
	\lstset{language=Python}          % Set your language (you can change the language for each code-block optionally)
		
	%\thispagestyle{empty}
	
	\begin{abstract}
		Using the modular number system and quadratic residues to classify integers as prime or composite. Computationally performing the Fermat primality test and identifying Carmichael numbers. Calculating the Jacobi symbol. Using the Solovay-Strassen test and finding Euler-Jacobi pseudoprimes.
	\end{abstract}

	\maketitle
	
	\tableofcontents
	\newpage
	
	% ==========================================================================
	% Intro
	% ==========================================================================		
	
	\section{Introduction}
%		Machines have displayed a significant level of learning ability. But how well can a machine learn a primality test from the integer sequence associated with it?

		
	\subsection{Prime and Composite Numbers}
	
	\begin{definition}[Prime]\label{definition:prime}
		Let $p \in \Z$, $p > 1$. Then $p$ is prime if and only if for every $a, b \in \Z$, $p=ab$ implies $a=1$ or $b=1$. \cite{pommersheim}
	\end{definition}
	
	\begin{definition}[Composite]\label{definition:composite}
		Let $n \in \Z, n > 1$. Then $n$ is composite if and only if there exists $a, b \in \Z$ such that $n=ab, 1<a,b<n$. \cite{pommersheim}
	\end{definition}

	\subsection{Divisibility}
		
	\begin{definition}[Divide \cite{pommersheim}]\label{definition:divide}
		Let $a,d \in \Z$. We say that $d$ divides $a$ if there exists $q \in \Z$ such that $a = qd$. We express this in symbols as $d \mid a$ (which is read ``$d$ divides $a$").
	\end{definition}
		
	\begin{prop}[Linear combination]\label{proposition:linear_combination}
		Let $d,m,n,x,y \in \Z$. If $d \mid x$ and $d \mid y$, then $d \mid mx + ny$.
	\end{prop}
	
	\begin{proof}
		Let $d,m,n,x,y \in \Z$ such that $d \mid x$ and $d \mid y$.
		Since $d \mid x$, then there exists $k \in \Z$ such that $x = kd$.
		Multiplying by $m$, we have  $mx = mkd$.
		Since $d \mid y$, then there exists $l \in \Z$ such that $y = ld$.
		Multiplying by $n$, we have  $ny = nld$.
		Adding both results yields $$ mx + ny = mkd + nld = (mk+nl)d.$$
		Therefore, since $(mk+nl) \in \Z$, we know $d \mid mx + ny$.
	\end{proof}
	
	\begin{definition}[Greatest common divisor]\label{definition:gcd}
		Let $a,b \in \Z$ be nonzero. The greatest common divisor of $a$ and $b$ is the largest $d \in \N$ for which $d \mid a$ and $d \mid b$ and is denoted $d = \gcd(a,b)$.
	\end{definition}
	
	\begin{example}
		The divisors of $42$ are $1,2,3,6,7,14,21,42$ and the divisors of $49$ are $1,7,49$. The largest number that appears in both of these lists is $7$. Thus, $7 = \gcd(42,49)$.
	\end{example}
	
	However, the greatest common divisor of some integers is $1$. For instance, the divisors of $25$ are $1,5,15$ and the divisors of $28$ are $1,2,4,7,14,28$. The largest number that appears in both lists is $1$. Such numbers are said to be relatively prime.
	
	\begin{prop}\label{proposition:smallest_possible_linear_combination}
		Let $a,b \in \Z$. Then greatest common divisor of $a$ and $b$ is equal to the smallest positive linear combination of $a$ and $b$.
	\end{prop}
	
	\begin{proof}
		We will use contradiction.
		Let $a,b \in \Z$, and suppose $e=ax+by$ is the smallest positive linear combination of $a$ and $b$.
		Let $d = \gcd(a,b)$. Then $d \mid a$ and $d \mid b$ implies that $d \mid ax + ay$ by 		\cref{proposition:linear_combination}.
		Since $d \mid ax+by$, then $d \mid e$, and so $d \leq e$.
		
		Now suppose $e \nmid a$. Thus, $a = qe + r$ for some $q \in \Z$ with $0 < r < e$. Then 
		$$r = a-qe = a-q(ax-by) = a-qax-qby = a(1-qx)- b(qy).$$
		This means that $r$ is a positive linear combination that is less than $e$, contradicting the fact that $e$ is the smallest positive linear combination of $a$ and $b$. Hence, $e \mid a$, and a similar argument can be used to show that $e \mid b$ as well. Thus, by \cref{proposition:linear_combination}, $e \mid d$ and $e \mid ax + by$, and so $e \leq d$.
		
		Since we have $d \leq e$ and $e \leq d$, it must be the case that $e=d$. Therefore, the greatest common divisor of $a$ and $b$ is equal to the smallest positive linear combination of $a$ and $b$.
	\end{proof}
	
	\begin{definition}[Relatively prime]\label{definition:relatively_prime}
		Let $a, b \in \Z$. We say $a$ and $b$ are relatively prime if $\gcd(a,b)=1$. 
	\end{definition}
	
	\begin{prop}\label{prop:gcd(a,p)=1}
		Let $a, p \in \Z$ such that $p$ is prime and $p \nmid a$, then $\gcd(a,p) = 1$.
	\end{prop}
	
	\begin{proof}
		We will use contradiction. 
		Let $a, p \in \Z$ such that $p$ is prime and $p \nmid a$. 
		Suppose $\gcd(a,p) \neq 1$, then $a = kd$ and $p = ld$ for some $d,k,l \in \Z$ with $d \neq 1$.
		Since $p$ is prime and $d \neq 1$, then $l=1$ and $d=p$ by \cref{definition:prime}.
		But $a=kd=kp$ if and only if $p \mid a$, contradicting the fact that $p \nmid a$.
		Therefore, we must conclude that $\gcd(a,p) = 1$.
	\end{proof}
	

	
	% ==========================================================================
	% Congrunces
	% ==========================================================================	
	\section{The Set of All Congruence Classes}
	
%	\begin{lemma}\label{lemma:divisibility}
%		Let $a, d \in \Z: a, d \geq 1$. If $d | a$, then $1 \leq d \leq a$. \cite{pommersheim}
%	\end{lemma}
%	
%	\begin{proof}
%		Let $a, d \in \Z: a, d \geq 1$ such that $d | a$. By \cref{definition:divisibility}, $\exists q \in \Z : a = qd$ since $d | a$. Furthermore, it must be that $q \geq 1$ and $d \geq 1$ so that $qd = a \geq 1$. Hence, $1 \leq d \leq qd = a$. Therefore, $1 \leq d \leq a$.
%	\end{proof}
%	
	
	\begin{definition}[Congruent \cite{pommersheim}]\label{definition:congruent}
		Let $a,b,n \in \Z$ with $n > 0$. We say that $a$ is congruent to $b$ modulo $n$ if $n \mid (a - b)$, denoted $a \equiv b \pmod n$. 
	\end{definition}

	\begin{example}\label{example:congruent}
		Is it true that 34 is congruent to 144 modulo $10$? Subtracting 144 from 34, we have $34 - 144 = -110$. Now, does $10$ divide this difference? Yes, since $-110 = -11 \cdot 10$. Thus, $34 \equiv 144 \pmod{10}$.
	\end{example}
	
%	Notice that every integer that is a multiple of $-1$ is also a multiple of $1$. Thus for instance $-1 | 5 - 3$ since $5-3 = 2 = -2(-1)$ and $1 | 5 - 3$ since $5-3 = 2 = 2(1)$. Therefore, $5$ and $3$ are congruent modulo $-1$ and modulo $1$. The same is true for $-54$ and $54$ because $120 - 12 = 108 = -2(-54)$ and $120 - 12 = 108 = 2(54)$; that is, $120$ and $12$ are congruent modulo $-54$ and modulo $54$. Hence, we will 
%	
%	\begin{prop}
%		Congruence is an equivalence relation.
%	\end{prop}
%	
%	\begin{proof}
%		Let $a,b,c,n \in \Z, n>0$. Since $0 = 0 \cdot n$, we know $n \mid 0$. Then $n \mid 0 = n \mid a-a$ if and only if $a \equiv a \pmod n$. Thus, congruence is reflexive. To show that it is also symmetric, suppose $a \equiv b \pmod n$. Notice that $n \mid p \iff p = qn \iff -p = -qn \iff n \mid -p$. Hence, $n$ divides both $q$ and $-q$. Thus, $a \equiv b \pmod m \iff n \mid a - b \iff n \mid -(a-b) = n \mid b-a \iff b \equiv a \pmod n$, and so congruence is symmetric. Lastly, we must show transitivity. So, assume $a \equiv b \pmod n$ and $b \equiv c \pmod n$. This means that $n$ divides both $a-b$ and $b-c$. Then, by \cref{definition:divisibility}, $a-b = kn$ and $b-c = ln$ for some $k,l \in \Z$. Adding these we have $a-c = a-b+b-c = (a-b)+(b-c) = kn + ln = (k+l)n$; that is, $a-c = (k+l)n$, which implies that $n$ divides $a-c$. Therefore, $a \equiv c \pmod n$, and so we have proven transitivity.
%	\end{proof}
	
	\begin{prop}\label{proposition:equivalent_conditions_for_congruent}
		Let $a, b, n \in \Z$. Then the following conditions are all equivalent.
		\begin{enumerate}
			\item $a = b + kn$ for some $k \in \Z$;
			\item $n \mid a - b$;
			\item $a \equiv b \pmod n$.
		\end{enumerate}		
	\end{prop}
	
	\begin{proof}
		Let $a, b, n \in \Z$.
		First suppose condition (1).
		Then $a = b + kn$ implies $a-b = kn$.
		By the definition of divide, it follows that $n \mid a - b$.
		Thus, condition (1) implies condition (2).
		
		Now suppose condition (2).
		Then, by the definition of congruent, $n \mid a - b$ implies $a \equiv b \pmod n$.
		Therefore, condition (2) implies condition (3).
		
		Finally, we would like to show that condition (3) implies condition (1).
		So, we assume $a \equiv b \pmod n$.
		Then, by the definition of congruent, we know $n \mid a - b$.
		If follows that there exists $k \in \Z$ such that $a-b = kn$ by the definition of divide.
		Therefore, condition (3) implies condition (1).

	\end{proof}
	
	% Which integers are also congruent to $144$?
	
%	\begin{theorem}[Equivalent Conditions for Congruence]\label{theorem:equivalent_conditions_for_congruence}
%		Let $a, b \in \Z, n \in \N$. The following statements are equivalent:
%		\begin{enumerate}
%			\item $a \equiv b \pmod b$
%			\item $n \mid (a - b)$
%			\item $\exists k \in \Z$ such that $a = b + kn$
%			\item when $a$ and $b$ are divided by $n$, they leave the same remainder.
%		\end{enumerate} \cite{pommersheim}
%	\end{theorem}
%	
	\begin{definition}[Congruence Class]\label{definition:congruence_class}
%		Let $n \in \N$, and let $b \in \Z$. Then the \textbf{congruence class} of $b$ modulo $n$ is the set of all integers congruent to $b$ modulo $n$: $$ \bar{b} := \{ x \in \Z : x \equiv b \pmod n \}. $$ The congruence class of $b$ modulo $n$ may also be defined as $$ \bar{b} := \{ b + kn : k \in \Z \}. $$
		Let $a,n \in \Z$ with $n > 0$. We define the congruence class of $a$ modulo $n$ as the set of all integers congruent to $a$ modulo $n$; that is, $$\bar{a} := \{ x \in \Z : x \equiv a \pmod n \}.$$% $$ \bar{a} := \{ a + kn : k \in \Z \}.$$
	\end{definition}
	
	\begin{definition}[$\Z_n$]\label{Definition: Zn}
		Let $n > 0$ be any integer. We define $\Z_n$ to be the set of all congruence classes modulo $n$, i.e. $$ \Z_n := \{ \bar{0}, \bar{1}, \bar{2}, \ldots, \overline{n - 1} \}.$$ %where $\bar{a}$ is defined, for any $a \in \Z$, to be the congruence class of $a$ modulo $n$: $$ \bar{a} := \{ a + kn : k \in \Z \}. $$
		
%		 The operations of addition and multiplication in $\Z_n$ are defined by 
%		\begin{align*}
%		\bar{a} + \bar{b} &= \overline{a + b}, \text{and} \\
%		\bar{a} \cdot \long\bar{b} &= \overline{ab}.
%		\end{align*} 
%		
%		If $k \in \N$, we can define $\bar{a}^k$ to be the product of $\bar{a}$ with itself $k$ times:$$\bar{a}^k = \underbrace{\bar{a} \cdot \bar{a} \cdot \bar{a} \cdot \ldots \cdot \bar{a}}_{k \text{ times}}.$$ We also allow an exponent of 0, defining $\bar{a}^0 = \bar{1}$. We can even define negative exponents, provided that $\bar{a}$ has a multiplication inverse: For any $k\in \N$, we define $\bar{a}^{-k}$ to be the product of $\bar{a}^{-1}$ with itself $k$ times. \cite{pommersheim}
	\end{definition}

	\begin{prop}[$\Z_n \neq \emptyset$]\label{proposition:Zn_is_nonempty}
		Let $n > 0$ be any integer. Then $\Z_n \neq \emptyset$.
	\end{prop}
	
	\begin{proof}
		Let $n > 0$ be any integer.
		Then for $k=1 \in \Z$, $n = 0 + kn$ implies $n \equiv 0 \pmod n$ by \cref{proposition:equivalent_conditions_for_congruent}.
		By the definition of congruence classes, it follows that $n \equiv 0 \pmod n$ is a member of $\bar{0} \in \Z_n$.
		Therefore, $\Z_n \neq \emptyset$.
	\end{proof}
	
	\begin{prop}\label{proposition:equal_congruence_classes}
		Let $\bar{a}, \bar{b} \in \Z_n$.
		Then $\bar{a} = \bar{b}$ if and only if $a \equiv b \pmod n$.
	\end{prop}	
	
	\begin{proof}
		Let $\bar{a}, \bar{b} \in \Z_n$. First, we would like to show that if $\bar{a} = \bar{b}$, then $a \equiv b \pmod n$.
		Suppose $\bar{a} = \bar{b}$, and let $x \in \bar{a}$.
		Thus, $x = a + kn$ for some $k \in \Z$ by the definition of congruence classes.
		Since $\bar{a}$ and $\bar{b}$ are equal sets, then it must be that $x \in \bar{b}$, and so $x = b + ln$ for some $l \in \Z$.
		Then
		\begin{align*}
			a+kn &= b+ln \\
			a-b &= ln-kn \\
			a-b &= (l-k)n.
		\end{align*}
		Since $l-k$ is an integer, we conclude that $n \mid a-b$.
		Thus, by the definition of congruent, $a \equiv b \pmod n$.
		
		Conversely, suppose $a \equiv b \pmod n$.
		Then $n \mid a - b$, which means that $a-b = mn$ for some $m\in \Z$.
		Since $m$ is an integer, it can be written as the sum of two integers, $p$ and $q$.
		Hence,
		\begin{align*}
			a-b &= mn \\
			a-b &= (p+q)n \\
			a-b &= pn+qn \\
			a-pn &= b + qn \\
			a + (-p)n &= b + qn
		\end{align*}
		Since the left-hand side of this equality is an arbitrary element of $\bar{a}$ and the right-hand side is an arbitrary element of $\bar{b}$, we conclude that these congruence classes must be equal.
		Therefore, $\bar{a} = \bar{b}$.
	\end{proof}
	
	
	We now define two basic operations $\Z_n$: addition and multiplication.
	
%	\begin{definition}[Addition and multiplication on $\Z_n$]\label{Definition: Addition and multiplication on Zn}
%		If $\bar{a}, \bar{b} \in \Z_n$, we define their sum $$\bar{a} + \bar{b} := \overline{a + b}, $$ and their product $$ \bar{a} \cdot \bar{b} := \overline{a \cdot b}.$$
%		We refer to this addition as addition on $\Z_n$ or addition modulo $n$. Similarly, this multiplication is called multiplication on $\Z_n$ or multiplication modulo $n$.
%	\end{definition}

	\begin{definition}[Addition on $\Z_n$]\label{definition:addition on Zn}
		Let $\bar{a}, \bar{b} \in \Z_n$.
		Then addition on $\Z_n$ is the function $f: \Z_n \times \Z_n \to \Z_n$ defined by $$f[(\bar{a}, \bar{b})] := \overline{a + b},$$
		and denoted $$ \bar{a} + \bar{b} := \overline{a + b}.$$
		We refer to this addition as addition on $\Z_n$ or addition modulo $n$.
	\end{definition}
	
	\begin{definition}[Multiplication on $\Z_n$]\label{definition:multiplication on Zn}
		Let $\bar{a}, \bar{b} \in \Z_n$.
		Then multiplication on $\Z_n$ is the function $f: \Z_n \times \Z_n \to \Z_n$ defined by $$f[(\bar{a}, \bar{b})] := \overline{a \cdot b},$$
		and denoted $$ \bar{a} \cdot \bar{b} := \overline{a \cdot b}.$$
		We refer to this multiplication as multiplication on $\Z_n$ or multiplication modulo $n$, and use the abbreviation $ab$ for $a \cdot b$.
	\end{definition}
	
%	
%	\begin{lemma}\label{Lemma: Addition and multiplication are closed and well-defined}
%		Addition and multiplication are closed and well-defined on $\Z_n$.
%	\end{lemma}
%	
%	\begin{proof}
%		First, we must show that addition and multiplication on $\Z_n$ are closed. By \cref{Definition: Addition and multiplication on Zn}, we know that for any $\bar{a}, \bar{b} \in \Z_n$, we have $\bar{a} + \bar{b} = \overline{a + b} \in \Z_n$ and $\bar{a} \cdot \bar{b} = \overline{a \cdot b} \in \Z_n$. Thus, addition and multiplication are closed. 
%		
%		Now we will show that this addition is well-defined. Suppose $\bar{a},\bar{b},\bar{c},\bar{d} \in \Z_n$, and that $\bar{a} = \bar{c}$ and $\bar{b} = \bar{d}$. Then, by \cref{proposition:a=n+kn}, $a - c = pn$ and $b - d = qn$ for some $p,q \in \Z$. Adding both equations, we have $a - c + b - d = pn + qn$, which can we rewritten as $(a + b) - (c + d) = (p+q)n$. This implies that $a + b \equiv c + d \pmod n$ and so $\overline{a + b} = \overline{c + d}$. Hence, this addition is well-defined. 
%		
%		
%		On the other hand, if we multiply $a-c = pn$ by $b$ and $b-d = qn$ by $c$, we have $ab-bc=bpn$ and $bc-cd=cqn$. Adding both equations tells us $ab-cd = ab-bc+bc-cd=bpn + cqn = (bp + cq)n$. Since $bp + cq \in \Z$, we have shown that $ab \equiv cd \pmod n$ and so $\overline{a \cdot b} = \overline{c \cdot d}$ as required. Thus, multiplication modulo $n$ is well-defined.
%	\end{proof}
%	
	
	
%	\begin{axioms}[Field \cite{mathworld:field_axioms}]\label{Field axioms}
%		A field is a set $F$ together with two closed and well-defined operations, namely addition and multiplication, that satisfy the following axioms. Let $a,b,c \in F$, then
%		\begin{table}[h]
%			\centering
%			%\caption{My caption}
%			%\label{my-label}
%			\begin{tabular}{|l|l|l|} \hline
%				Name           & Addition 					& Multiplication 						\\ \hline
%				Associativity  & (1) $(a+b)+c = a+(b+c)$	& (2) $(a\cdot b) c = a (b \cdot c)$	\\ \hline
%				Commutativity  & (3) $a + b = b + a$		& (4) $a \cdot b = b \cdot a$			\\ \hline
%				Distributivity & (5) $a(b + c) = ab + ac$	& (6) $(a + b)c = ac + bc$				\\ \hline
%				Identity       & (7) $a + 0 = a = 0 + a$	& (8) $a \cdot 1 = a = 1 \cdot a$		\\ \hline
%				Inverses       & (9) $a+(-a) = 0 = (-a)+a$	& (10) $a \cdot a^{-1} = 1 = a^{-1} \cdot a$ if $a \neq 0$ \\ \hline
%			\end{tabular}
%		\end{table}		
%	\end{axioms}
%	
%	\begin{proof}
%		Let $p$ be prime.
%		Then $p - 0 = 1p$ if and only if $p \equiv 0 \pmod p$; that is, $\bar{0} \in \Z_p$, and so $\Z_p$ is nonempty.
%		Since $1 - 1 = 0 \cdot p$, then $1 \equiv 1 \pmod p$. Hence, we also have that $\bar{1} \in \Z_p$.
%		Furthermore, by \cref{Lemma: Addition and multiplication are closed and well-defined}, we know that addition and multiplication are closed and well-defined on $\Z_p$, and so now we must show that these operations satisfy the field axioms.
%		
%		Let $\bar{a}, \bar{b}, \bar{c} \in \Z_n$ be arbitrary. We can exploit the associativity of integer addition and multiplication in order to show that addition and multiplication on $\Z_n$ are also associative. Thus,
%		\begin{enumerate}
%			\item $(\bar{a} + \bar{b}) + \bar{c} = \overline{a + b}  + \bar{c} = \overline{(a + b) + c} = \overline{a + (b + c)} = \bar{a} + \overline{b + c} = \bar{a} + (\bar{b} + \bar{c})$, and
%			\item $(\bar{a} \cdot \bar{b}) \cdot \bar{c} = \overline{a \cdot b} \cdot \bar{c} = \overline{(a \cdot b) \cdot c} = \overline{a \cdot (b \cdot c)} = \bar{a} \cdot \overline{b \cdot c} = \bar{a} \cdot (\bar{b} \cdot \bar{c})$.
%		\end{enumerate}
%		Likewise with commutativity, 
%		\begin{enumerate}
%			\item[(3)] $\bar{a} + \bar{b} = \overline{a + b} = \overline{b + a} = \bar{b} + \bar{a}$, and
%			\item[(4)] $\bar{a} \cdot \bar{b} = \overline{a \cdot b} = \overline{b \cdot a} = \bar{b} \cdot \bar{a}$.
%		\end{enumerate} 
%		Furthermore, since multiplication distributes over addition on the integers, we also have
%		
%		\begin{enumerate}
%			\item[(5)] $\bar{a} \cdot (\bar{b}+\bar{c}) = \bar{a} \cdot \overline{b+c} = \overline{a (b+c)} = \overline{a \cdot b+ac} = \overline{a \cdot b} + \overline{a \cdot c} = \bar{a} \cdot \bar{b} + \bar{a} \cdot \bar{c}$, and 
%			\item[(6)] $(\bar{a}+\bar{b}) \cdot \bar{c} = \overline{a + b} \cdot \bar{c} = \overline{(a + b)c} = \overline{a \cdot c + b \cdot c} = \overline{a \cdot c} + \overline{b \cdot c} = \bar{a} \cdot \bar{c} + \bar{b} \cdot \bar{c}$.
%		\end{enumerate}
%		Thus, know that addition and multiplication on $\Z_p$ are associative, commutative, and distributive.
%		
%		As for additive and multiplicative identities, since $\bar{0}, \bar{1} \in \Z$, we have 
%		\begin{enumerate}
%			\item[(7)] $\bar{a} + \bar{0} = \overline{a + 0} = \bar{a} = \overline{0 + a} = \bar{a} + \bar{0}$, and
%			\item[(8)] $\bar{a} \cdot \bar{1} = \overline{a \cdot 1} = \bar{1} = \overline{1 \cdot a} = \bar{1} \cdot \bar{a}$.
%		\end{enumerate}
%		Hence, in $\Z_p$, $\bar{0}$ is the additive identity and $\bar{1}$ is the multiplicative identity.
%		
%		\begin{enumerate}
%	
%			\item[(9)] To show that every element of $\Z_p$ has an additive inverse, let $\bar{d} = \overline{p + (-a)}$. Since $-a,p \in \Z$ and the integers are closed under addition, we have $p + (-a) \in \Z$, and so $p + (-a)$ belongs in one of the congruence classes of $\Z_p$. Thus, $\bar{d} \in \Z_p$. Then  $$\bar{a} + \bar{d} = \overline{a + p + (-a)} = \overline{a + (-a) + p} = \bar{p} = \overline{p +(-a) + a} = \overline{p + (-a)} + \bar{a} = \bar{d} + \bar{a}.$$ Because $p = 0 + kp : k \in \Z$, we know $\bar{p} = \bar{0}$. Therefore, $\bar{a} + \bar{d} = \bar{0} = \bar{d} + \bar{a}$. Since $\bar{d}$ is the additive inverse of $\bar{a}$, we say $\bar{d} = -\bar{a}$ and use the short hand $\bar{a} - \bar{a}$ for $\bar{a} + (-\bar{a})$.
%			
%			\item[(10)] Finally, we would like to know if every nonzero element in $\Z_p$ has a multiplicative inverse.
%			So assume $\bar{a} \neq \bar{0}$. 
%			Thus, $\gcd(a,p)=1$ by \cref{lemma:all a in Zp relatively prime}.
%			Then, 
%			\begin{align*}
%				ax+py &= 1 & \text{(by \cref{proposition: gcd is smallest possible linear combination})} \\
%				ax &= 1 + (-y)p & \\
%				ax &\equiv 1 \pmod p & \text{(by \cref{proposition:a=n+kn})}.
%			%	\overline{a \cdot x} &= 1 & \\
%			%	\bar{a} \cdot \bar{x} &= 1.
%			\end{align*}
%			Thus, $\overline{a \cdot x} = \bar{1}$, and so $\bar{a} \cdot \bar{x} = \bar{1}$. Hence, $\bar{x}$ is a multiplicative inverse of $\bar{a}$, and we say $\bar{x} = \bar{a}^{-1}$. Therefore, $\bar{a} \cdot \bar{x} = \bar{a} \cdot \bar{a}^{-1} = \bar{1} = \bar{a} \cdot \bar{a}^{-1} = \overline{a \cdot a^{-1}} = \overline{a^{-1} \cdot a} = \bar{a}^{-1} \cdot \bar{a}$, and so $\bar{a} \cdot \bar{a}^{-1} = \bar{1} = \bar{a}^{-1} \cdot \bar{a}$ as required.
%		\end{enumerate}
%		Therefore, if $p$ is prime, then $\Z_p$ satisfies the field axioms.
%		
%	\end{proof}
%		
%	\begin{lemma}\label{lemma:equivalences_in_Z_n}
%		Let $\bar{a}, \bar{b} \in \Z_n$. If $\bar{a} = \bar{b}$, then $a = b + kn$ for some $k \in \Z$.
%	\end{lemma}
%	
%	\begin{proof}
%		Let $\bar{a}, \bar{b} \in \Z_n$ such that $\bar{a} = \bar{b}$. By \cref{definition:Z_n}, $ \bar{a} = a + in = b + jn = \bar{b}$ for some $i,j \in \Z$; that is, $a = b + jn - in = b + (j - i)n = b + kn$, where $j - i = k \in \Z$.
%	\end{proof}
%	
%	\begin{definition}
%		A ring $D$ is called an \textbf{integral domain} provided 
%		\begin{enumerate}
%			\item $D$ is a commutative ring,
%			\item 0 and 1 name different elements of $D$,
%			\item If $a,b \in D$ and $a \cdot b = 0$, then either $a=0$ or $b=0$.
%		\end{enumerate} \cite{mcnulty}
%	\end{definition}
%	
%	\begin{cor}
%		$\Z_n$ is a ring with identity.	
%	\end{cor}
%	
%	\begin{proof}
%		Let $\bar{a} \in \Z_n$. Since $1 \in \Z$, we know that $\bar{1} = 1 + ln \in \Z_n$ for some $l \in \Z$ and $\bar{1} \neq \bar{0}$. Then, 
%		$$ \bar{1} \cdot \bar{a} = \overline{1 \cdot a} = \bar{a} = \overline{a \cdot 1} = \bar{a} \cdot \bar{1}.$$ According to \cref{definition:ring}, the fact that $\bar{1} \cdot \bar{a} = \bar{a} \cdot \bar{1} = \bar{a}$ tells us that $\Z_n$ has identity. Therefore, $\Z_n$ is a ring with identity.
%	\end{proof}
%
%	\begin{cor}
%		$\Z_n$ is a commutative ring.	
%	\end{cor}
%	
%	\begin{proof}
%		Let $\bar{a}, \bar{b} \in \Z_n$. Then, since integer multiplication is commutative, we have $\bar{a} \cdot \bar{b} = \overline{a \cdot b} = \overline{b \cdot a} = \bar{b} \cdot \bar{a}$ as required. Therefore, $\Z_n$ is a commutative ring.
%	\end{proof}
%		
%	\begin{cor}\label{corollary:ab=0_then_a=0_or_b=0}
%		 For every $\bar{a}, \bar{b} \in \Z_n$ such that $\bar{a} \cdot \bar{b} = \bar{0}$, either $\bar{a} = \bar{0}$ or $\bar{b} = \bar{0}$.	
%	\end{cor}
%	
%	\begin{proof}
%		 Let $\bar{a}, \bar{b} \in \Z_n$ such that $\bar{a} \cdot \bar{b} = \bar{0}$. Suppose $\bar{a} \neq \bar{0}$ and $\bar{b} \neq \bar{0}$. Thus $a,b \in \Z $ such that $a \neq 0$ and $b \neq 0$. But $\bar{a} \cdot \bar{b} = \overline{ab} \neq \bar{0}$ since the product of two nonzero integers cannot equal 0; which contradicts $\bar{a} \cdot \bar{b} = \bar{0}$. Therefore, we conclude that either $\bar{a} = \bar{0}$ or $\bar{b} = \bar{0}$.
%	\end{proof}
	
	%\subsection{Fundamental Theorems of Congruences}
	%\subsection{Fermat's Little Theorem}
	
	
%	\begin{theorem}[Wilson's Theorem]\label{theorem:wilsons_theorem}
%		Let $p \in \Z$ be prime. Then in $\Z_p$, $$ \bar{1} \cdot \bar{2} \cdot \bar{3} \cdot \ldots \cdot (\overline{p-1}) = \overline{-1}. $$
%	\end{theorem}
	

	%		
	%	\begin{theorem}[Fermat's Little Theorem]\label{theorem:fermats_little_theorem}
	%		Let $p \in \N$ be prime, and let $\bar{a} \in \Z_p : \bar{a} \neq \bar{0}$. Then $$ \bar{a}^{p-1} = \bar{1}. $$\cite{pommersheim}
	%	\end{theorem}
	%	
%	\begin{theorem}[Euler's Theorem]\label{theorem:eulers_theorem}
%		Let $n,a \in \Z$ such that $n > 0$ and $\gcd(a,n) = 1$. Then $a^{\phi(n)} = 1 \pmod n$. \cite{koshy}
%	\end{theorem}	
%		

	% ==========================================================================
	% Algebraic Structures
	% ==========================================================================		
	
	\section{Algebraic Structures with $\Z_n$}

	
	\begin{definition}[Binary operation]\label{definition:bianry operation}
		Let $S$ be a set. We define a binary operation on $S$ to be a function $f:S \times S \to S$ that assigns to each pair $(a,b) \in S \times S$ a unique element $a \circ b \in S$.
		
		A binary operation $\circ$ with the property that $(a \circ b) \circ c = a \circ (b \circ c)$ for all $a, b, c \in S$ is called associative.
	\end{definition}
		
	% Please add the following required packages to your document preamble:
	% \usepackage{booktabs}
	\begin{table}[]
		\centering
		\caption{Group-like Algebraic Structures}
		\label{table:group-like_algebraic_structures}
		\begin{tabular}{@{}cccccc@{}}
			\toprule
			& \textbf{\begin{tabular}[c]{@{}c@{}}Binary \\ Operation\end{tabular}} & \textbf{Associativity} & \textbf{Identity} & \textbf{Inverses} & \textbf{Commutativity} \\ \midrule
			\textbf{Magma}         & Required                                                             & Unneeded               & Unneeded          & Unneeded          & Unneeded               \\
			\textbf{Semigroup}     & Required                                                             & Required               & Unneeded          & Unneeded          & Unneeded               \\
			\textbf{Monoid}        & Required                                                             & Required               & Required          & Unneeded          & Unneeded               \\
			\textbf{Group}         & Required                                                             & Required               & Required          & Required          & Unneeded               \\
			\textbf{Abelian Group} & Required                                                             & Required               & Required          & Required          & Required               \\ \bottomrule
		\end{tabular}
	\end{table}	

	% Please add the following required packages to your document preamble:
	% \usepackage{booktabs}
	% \usepackage[table,xcdraw]{xcolor}
	% If you use beamer only pass "xcolor=table" option, i.e. \documentclass[xcolor=table]{beamer}
%	\begin{table}[]
%		\centering
%		\caption{My caption}
%		\label{my-label}
%		\begin{tabular}{@{}cccccc@{}}
%			\toprule
%			& \textbf{\begin{tabular}[c]{@{}c@{}}Binary \\ Operation\end{tabular}} & \textbf{Associativity}           & \textbf{Identity}                & \textbf{Inverses}                & \textbf{Commutativity}           \\ \midrule
%			\textbf{Magma}         & \cellcolor[HTML]{9AFF99}Required                                     & \cellcolor[HTML]{FFCCC9}Unneeded & \cellcolor[HTML]{FFCCC9}Unneeded & \cellcolor[HTML]{FFCCC9}Unneeded & \cellcolor[HTML]{FFCCC9}Unneeded \\
%			\textbf{Semigroup}     & \cellcolor[HTML]{9AFF99}Required                                     & \cellcolor[HTML]{9AFF99}Required & \cellcolor[HTML]{FFCCC9}Unneeded & \cellcolor[HTML]{FFCCC9}Unneeded & \cellcolor[HTML]{FFCCC9}Unneeded \\
%			\textbf{Monoid}        & \cellcolor[HTML]{9AFF99}Required                                     & \cellcolor[HTML]{9AFF99}Required & \cellcolor[HTML]{9AFF99}Required & \cellcolor[HTML]{FFCCC9}Unneeded & \cellcolor[HTML]{FFCCC9}Unneeded \\
%			\textbf{Group}         & \cellcolor[HTML]{9AFF99}Required                                     & \cellcolor[HTML]{9AFF99}Required & \cellcolor[HTML]{9AFF99}Required & \cellcolor[HTML]{9AFF99}Required & \cellcolor[HTML]{FFCCC9}Unneeded \\
%			\textbf{Abelian Group} & \cellcolor[HTML]{9AFF99}Required                                     & \cellcolor[HTML]{9AFF99}Required & \cellcolor[HTML]{9AFF99}Required & \cellcolor[HTML]{9AFF99}Required & \cellcolor[HTML]{9AFF99}Required \\ \bottomrule
%		\end{tabular}
%	\end{table}
	
	\subsection{Magma}
	\begin{definition}[Magma]\label{definition:magma}
		A magma is a set $M$ together with a binary operation $\circ$ such that for all $a,b \in M$, the unique result of the operation $a \circ b$ is also in $M$.
	\end{definition}
	
	\begin{prop}\label{proposition:Zn_is_a_magma_under_addition_modulo_n}
		Let $n > 0$ be any integer. Then $\Z_n$ is a magma under addition modulo $n$.
	\end{prop}
	
	\begin{proof}
		Let $n > 0$ be any integer.
		By \cref{proposition:Zn_is_nonempty}, we know that $\Z_n \neq \emptyset$.
		So, let $\bar{a}, \bar{b} \in \Z_n$.
		First, we must show that this addition is closed. By the definition of addition on $\Z_n$, we know that $\bar{a} + \bar{b} = \overline{a + b} \in \Z_n$, and so $\Z_n$ is closed under this operation.
		
		Next, we must confirm that this addition assigns a unique element $\bar{a} + \bar{b} \in \Z_n$ to each pair $\bar{a},\bar{b} \in \Z_n$.		
		Let $\bar{a}^\prime,\bar{b}^\prime \in \Z_n$ such that $\bar{a} = \bar{a}^\prime$ and $\bar{b} = \bar{b}^\prime$. Then, by \cref{proposition:equivalent_conditions_for_congruent}, $a = a^\prime + kn$ and $b = b^\prime + ln$ for some $k,l \in \Z$. 
		Adding both equations, we have $a + b = a^\prime + kn + b^\prime + ln$; which can rewritten as $(a + b) = (a^\prime + b^\prime) + mn$ with $m = (k + l) \in \Z$.
		Again by \cref{proposition:equivalent_conditions_for_congruent}, we conclude that $a + b \equiv a^\prime + b^\prime \pmod n$, and so $\bar{a} + \bar{b} = \overline{a + b} = \overline{a^\prime + b^\prime} = \bar{a}^\prime + \bar{b}^\prime$.
		Thus, this addition assigns to each pair $(\bar{a},\bar{b}) \in \Z_n \times \Z_n$ a unique element $\bar{a} + \bar{b} \in \Z_n$.
		Therefore, $\Z_n$ is a magma under addition modulo $n$.
	\end{proof}

	\begin{prop}\label{proposition:Zn_is_a_magma_under_multiplication_modulo_n}
		Let $n > 0$ be any integer. Then $\Z_n$ is a magma under multiplication modulo $n$.
	\end{prop}

	\begin{proof}
		Let $n > 0$ be any integer.
		Then $\Z_n \neq \emptyset$ by \cref{proposition:Zn_is_nonempty}.
		Let $\bar{a}, \bar{b} \in \Z_n$.
		By the definition of this multiplication, we know $\bar{a} \cdot \bar{b} = \overline{a \cdot b} \in \Z_n$, and so the result of this multiplication is also in $\Z_n$.
		
		Now, we would like to show that this result is unique.
		So, let $\bar{a}^\prime, \bar{b}^\prime \in \Z_n$ with $\bar{a} = \bar{a}^\prime$ and $\bar{b} = \bar{b}^\prime$.
		Thus, by \cref{proposition:equal_congruence_classes}, we have $a \equiv a^\prime \pmod n$ and $b \equiv b^\prime \pmod n$.
		It follows that $n$ divides both $a-a^\prime$ and $b-b^\prime$, and so $a-a^\prime = kn$ and $b-b^\prime = ln$ for some $k,l \in \Z$.
		Multiplying the first equality by $b$ and the second by $a^\prime$, we obtain $ba-ba^\prime = bkn$ and $a^\prime b-a^\prime b^\prime = a^\prime ln$. Adding both equations  tells us
		\begin{align*}
		ba-ba^\prime + a^\prime b-a^\prime b^\prime &= bkn + a^\prime ln \\
		ab-a^\prime b + a^\prime b-a^\prime b^\prime &= (bk + a^\prime l)n \\
		ab -a^\prime b^\prime &= mn,
		\end{align*}
		with $m = (bk + a^\prime l) \in \Z$.
		Hence, $n \mid ab -a^\prime b^\prime$. %, and so $ab \equiv a^\prime b^\prime \pmod n$.
		By \cref{proposition:equivalent_conditions_for_congruent}, it follows that $\overline{ab} = \overline{a^\prime b^\prime}$, and so $\bar{a} \cdot \bar{b} = \overline{ab} = \overline{a^\prime b^\prime} = \bar{a}^\prime \cdot \bar{b}^\prime$.
		Thus, each pair $(\bar{a},\bar{b}) \in \Z_n \times \Z_n$ is mapped to a unique element $\bar{a} \cdot \bar{b} \in \Z_n$.
		Therefore, $\Z_n$ is a magma under multiplication modulo $n$.
	\end{proof}

	
	\subsection{Semigroup}
	
	\begin{definition}
		A semigroup $S$ is an associative magma; that is, $(a \circ b) \circ c = a \circ (b \circ c)$ for all $a,b,c \in S$.
	\end{definition}

	\begin{prop}\label{proposition:Zn_is_a_semigroup_under_addition_modulo_n}
		Let $n > 0$ be any integer. Then $\Z_n$ is a semigroup under addition modulo $n$.
	\end{prop}
	
	\begin{proof}
		Let $n > 0$ be any integer.
		According to \cref{proposition:Zn_is_a_semigroup_under_addition_modulo_n}, $\Z_n$ is a magma under addition modulo $n$.
		So all we have to show is this addition is associative.
		So, let $\bar{a}, \bar{b}, \bar{c} \in \Z_n$.
		Then, by the associativity of addition on the integers, we have $$(\bar{a} + \bar{b}) + \bar{c} = \overline{a + b}  + \bar{c} = \overline{(a + b) + c} = \overline{a + (b + c)} = \bar{a} + \overline{b + c} = \bar{a} + (\bar{b} + \bar{c}).$$
		Thus, $\Z_n$ is an associative magma, and so we conclude that $\Z_n$ is a semigroup under addition modulo $n$.
	\end{proof}
	
	\begin{prop}\label{proposition:Zn_is_a_semigroup_under_multiplication_modulo_n}
		Let $n > 0$ be any integer. Then $\Z_n$ is a semigroup under multiplication modulo $n$.
	\end{prop}

	\begin{proof}
		Let $n > 0$ be any integer.
		We know that, under multiplication modulo $n$, $\Z_n$ is a magma.
		To show that multiplication on $\Z_n$ is associative, let $\bar{a}, \bar{b}, \bar{c} \in \Z_n$.
		Then, by the associativity of integer multiplication, we have 
		\begin{align*}
		(\bar{a} \cdot \bar{b}) \cdot \bar{c} 
		&= \overline{a \cdot b} \cdot \bar{c} \\
		&= \overline{(a \cdot b) \cdot c} \\
		&= \overline{a \cdot (b \cdot c)} \\
		&= \bar{a} \cdot \overline{b \cdot c} \\
		&= \bar{a} \cdot (\bar{b} \cdot \bar{c}).
		\end{align*}
		Therefore, multiplication on $\Z_n$ is associative, and so $\Z_n$ is a semigroup under this operation.		
	\end{proof}


	\subsection{Monoid}

	\begin{definition}
		A monoid $M$ is a semigroup that has an identity element $e \in M$ such that $e \circ a = a \circ e = a$ for all $a \in M$.
	\end{definition}	

	\begin{prop}\label{proposition:Zn_is_a_monoid_under_addition}
		Let $n>0$ be any integer. Then the set $\Z_n$ forms a monoid under addition modulo $n$.
	\end{prop}
	
	\begin{proof}
		Let $n>0$ be any integer.
		By \cref{proposition:Zn_is_a_semigroup_under_addition_modulo_n}, $\Z_n$ is a semigroup under addition modulo $n$.
		Now we must also show that $\Z_n$ contains an identity element under this operation.
		%Since $n = 0 + kn$ for $k = 1 \in \Z$, we conclude that $n \equiv 0 \pmod n$ by \cref{proposition:equivalent_conditions_for_congruent}.
		%Thus, according to the definition of congruence classes, $\bar{0} \in \Z_n$.
		We saw in our proof of \cref{proposition:Zn_is_nonempty} that $\bar{0} \in \Z_n$.
		Let $\bar{a} \in \Z_n$ be arbitrary.
		Then $\bar{a} + \bar{0} = \overline{a + 0} = \bar{a} = \overline{0 + a} = \bar{0} + \bar{a}$; that is, $\bar{a} + \bar{0} =  \bar{a} = \bar{0} + \bar{a}$.
		Hence, $\bar{0} \in \Z_n$ is an identity element under this addition for all $\bar{a} \in \Z_n$.
		Therefore, we conclude that the set $\Z_n$ forms a monoid under addition modulo $n$.
	\end{proof}
		
	\begin{prop}\label{proposition:Zn_is_a_monoid_under_multiplication}
		Let $n>0$ be any integer. Then the set $\Z_n$ forms a monoid under multiplication modulo $n$.
	\end{prop}
	
	\begin{proof}
		Let $n>0$ be any integer.
		We already know that $\Z_n$ is a semigroup under multiplication modulo $n$ according to \cref{proposition:Zn_is_a_semigroup_under_multiplication_modulo_n}.
		So, all we have left to prove is that $\Z_n$ has an identity element.
		Since $(n+1) = 1 + kn$ for $k=1 \in \Z$, we know $(n+1) \equiv 1 \pmod n$ by \cref{proposition:equivalent_conditions_for_congruent}; which is en element in $\bar{1} \in \Z_n$.
		Thus, $\bar{1} \in \Z_n$.
		% By \cref{proposition:Zn_is_nonempty}, we know $\Z_n \neq \emptyset$ since $\bar{1} \in \Z_n$.
		Let $\bar{a} \in \Z_n$.
		Then, by the commutativity of integer multiplication, $$\bar{1} \cdot \bar{a} = \overline{1 \cdot a} = \bar{a} = \overline{a \cdot 1} = \bar{a} \cdot \bar{1};$$ that is, $\bar{1} \cdot \bar{a} = \bar{a} = \bar{a} \cdot \bar{1}$.
		Thus, $\Z_n$ has an identity element, namely $\bar{1} \in \Z_n$.
		Therefore, $\Z_n$ forms a monoid under multiplication modulo $n$.
%		First we must show that $\Z_n \setminus \{ \bar{0} \}$ is nonempty. 
%		%Let $\bar{a} \in \Z_n \setminus \{ \bar{0} \}$.
%		Since $a+1 \in \Z$, then $a + 1 = (a+1) \cdot 1$ implies that $1 \mid a+1$, and so $a \equiv 1 \pmod n$.
%		Thus, by \cref{definition:congruence_class}, $\bar{a}$ is a congruence class modulo $n$, and thus belongs in the set of all congruence classes modulo $n$, namely $\Z_n$.
%		
%		By \cref{Lemma: Addition and multiplication are closed and well-defined}, we know $\Z_n$ is closed under multiplication modulo $n$. 
%		To show that this multiplication is associative, let $\bar{a}, \bar{b}, \bar{c} \in \Z_n$.
%		Then, by the associativity of integer multiplication, we have $$(\bar{a} \cdot \bar{b}) \cdot \bar{c} = \overline{a \cdot b} \cdot \bar{c} = \overline{(a \cdot b) \cdot c} = \overline{a \cdot (b \cdot c)} = \bar{a} \cdot \overline{b \cdot c} = \bar{a} \cdot (\bar{b} \cdot \bar{c}).$$
%		Thus, $\Z_n$ is a set that is closed under an associative binary operation, namely multiplication modulo $n$.
%		
%		Now we will prove that 
	\end{proof}
	
	
	\subsection{Group}
	
	\begin{definition}[Group]\label{definition:group}
		We define a group $G$ to be a monoid that contains an inverse element for each element $a \in G$, denoted by $a^{-1}$, such that $a \circ a^{-1} = a^{-1} \circ a = e$.
%		\begin{enumerate}
%			\item The law of composition is associative. That is, $$(a \circ b) \circ c = a \circ (b \circ c)$$ for $a,b,c \in G$.			
%			\item There exists an element $e \in G$, called the identity element, such that for any element, such that for any element $a \in G$ $$e \circ a = a = a \circ e.$$
%			\item For each element $a \in G$, there exists an inverse element in $G$, denoted by $a^{-1}$, such that $$a \circ a^{-1} = a^{-1} \circ a = e.$$
%		\end{enumerate}
		A group with the property that $a \circ b = b \circ a$ for $a,b \in G$ is called abelian.
	\end{definition}

	\begin{prop}[Cancellation laws]\label{proposition:cancellation_laws}
		Let $G$ be a group, and let $a,b,c \in G$. Then
		\begin{align*}
		b \circ a=c \circ a &\text{ implies } b=c \text{ (left cancellation)}, \text{ and} \\
		a \circ b=a \circ c &\text{ implies } b=c \text{ (right cancellation)}.
		\end{align*}
	\end{prop}
	
	\begin{proof}
		Let $G$ be a group, and let $a,b,c \in G$.
		Since $a \in G$, then $a^{-1} \in G$ such that $a \circ a^{-1} = a^{-1} \circ a = e$.
		First, suppose $b \circ a=c \circ a$. Then 
		\begin{align*}
		b \circ a &= c \circ a \\
		b \circ a \circ a^{-1} &= c \circ a \circ a^{-1} \\
		b \circ e &= c \circ e \\
		b &= c.
		\end{align*}
		Now suppose $a \circ b=a \circ c$. Then
		\begin{align*}
		a \circ b &= a \circ c \\
		a^{-1} \circ a \circ b &= a^{-1} \circ a \circ c \\
		e \circ b &= e \circ c \\
		b &= c.
		\end{align*}
		Therefore, $b \circ a = c \circ a$ implies $b=c$ and $a \circ b = a \circ c$ implies $b=c$.			
	\end{proof}
		
	\begin{prop}[Unique identity]\label{proposition:unique_indenty}
		Let $G$ be a group. Then the identity element $e \in G$ is unique.
	\end{prop}
	
	\begin{proof}
		Let $G$ be a group, let $e\in G$ be the identity element, and let $g \in G$.
		Suppose $e^\prime \in G$ is also an identity element.
		Thus, $e \circ g = g$ and $e^\prime \circ g = g$, and so $e \circ g = e^\prime \circ g$. 
		Then, by right cancellation, we have $e = e^\prime$. 
		Therefore, the identity element $e \in G$ is unique
	\end{proof}
	
	\begin{prop}[Unique inverse]\label{proposition:unique_inverse}
		Let $G$ be a group. Then $g^{-1}$ is unique.
	\end{prop}
	
	\begin{proof}
		Let $g, g^{-1}, g^{\prime -1} \in \Z_n$ with $g \circ g^{-1} = e$ and $g \circ g^{\prime -1} = e$. 
		Thus, $g \circ g^{-1} = g \circ g^{\prime -1}$.
		Then, $g^{-1} = g^{\prime -1}$ by left cancellation.
		Therefore, $g^{-1}$ is unique.
	\end{proof}
	
	
	\begin{definition}[Exponential notation]\label{definition:exponential_notation}
		Let $G$ be a group, and let $g \in G$. We first define $g^0 = e$. For $n \in \N, n>0$, we define 
		$$ g^n = \underbrace{g \circ g \circ \cdots \circ g}_{n \text{ times}}$$ and 
		$$ g^{-n} = \underbrace{g^{-1} \circ g^{-1} \circ \cdots \circ g^{-1}}_{n \text{ times}}.$$
	\end{definition}
	
	\begin{prop}
		Let $G$ be a group, and let $g,h \in G$. Then, for all $m,n \in \Z$,
		\begin{enumerate}
			\item $g^m \circ g^n = g^{m+n}$,
			\item $(g^m)^n = g^{mn}$, and 
			\item $(g \circ h)^n = (h^{-1} \circ g^{-1})^{-n}.$
		\end{enumerate}
	\end{prop}
	
	\begin{proof}
		Let $G$ be a group, let $g,h \in G$, and let $m,n \in \Z$.
		Then, by the definition of exponential notation and the associativity of the group operation, we have 
		\begin{align*}
			g^m \circ g^n &= \underbrace{(g \circ g \circ \cdots \circ g)}_{m \text{ times}} \circ \underbrace{(g \circ g \circ \cdots \circ g)}_{n \text{ times}} \\
			&= \underbrace{g \circ g \circ \cdots \circ g \circ g \circ g \circ \cdots \circ g}_{m + n \text{ times}} \\
			&= g^{m + n} 
		\end{align*}
		and 
		\begin{align*}
			(g^m)^n &= \underbrace{g^m \circ g^m \circ \cdots \circ g^m}_{n \text{ times}} \\
			&= \underbrace{\overbrace{(g \circ g \circ \cdots \circ g)}^{m \text{ times}} \circ \overbrace{(g \circ g \circ \cdots \circ g)}^{m \text{ times}} \circ \cdots \circ \overbrace{(g \circ g \circ \cdots \circ g)}^{m \text{ times}}}_{n \text{ times}} \\
			&= \underbrace{g \circ g \circ \cdots \circ g \circ g \circ g \circ \cdots \circ g \circ \cdots \circ g \circ g \circ \cdots \circ g}_{mn \text{ times}} \\
			&= g^{mn}
		\end{align*}
		for the first and second equalities.
		Before continuing, we observe that
		\begin{align*}
			(h^{-1} \circ g^{-1}) \circ (g \circ h)
			&= h^{-1} \circ g^{-1} \circ g \circ h & \text{(associativity of group operation)} \\
			&=	 h^{-1} \circ e \circ h \\
			&= h^{-1} \circ h \circ e & (\text{since } e \circ h = h \circ e) \\
			&= e \circ e \\
			&= e.
		\end{align*}
		Thus, $(h^{-1} \circ g^{-1})^{-1} = (g \circ h)$. 
		It follows that 
		\begin{align*}
			(h^{-1} \circ g^{-1})^{-n} &= \underbrace{(h^{-1} \circ g^{-1})^{-1} \circ (h^{-1} \circ g^{-1})^{-1} \circ \cdots \circ (h^{-1} \circ g^{-1})^{-1}}_{n \text{ times}} \\
			&= \underbrace{(g \circ h) \circ (g \circ h) \circ \cdots \circ (g \circ h)}_{n \text{ times}} \\
			&= (g \circ h)^n.
		\end{align*}

	\end{proof}
	
	%	\begin{definition}[Subgroup \cite{judson}]\label{definition: subgroup}
	%		Let $G$ be a group. We define a subgroup $H$ of $G$, denoted $H \leq G$, to be a subset $H \subseteq G$ such that if the group operation $\circ$ of $G$ is restricted to $H$, then $H$ is a group on its own right. 
	%	\end{definition}
	%	
	%	\begin{theorem}[Cyclic subgroups]\label{theorem: cyclic subgroups}
	%		Let $G$ a group, and let $a \in G$.
	%		Then the set $$\langle a \rangle = \{ a^k : k \in \Z \}$$ is a subgroup of $G$. Furthermore, $\langle a \rangle$ is the smallest subgroup of $G$ that contains $a$. 
	%	\end{theorem}
	%	
	%	\begin{proof}
	%		Let $G$ a group, and let $a \in G$.
	%	\end{proof}

	
	\begin{prop}\label{proposition:Zn_is_an_abelian_group_under_addition}
		Let $n>0$ be any integer. Then the set $\Z_n$ forms an abelian group under addition modulo $n$.
	\end{prop}
	
	\begin{proof}
		Let $n>0$ be any integer.
		According to \cref{proposition:Zn_is_a_monoid_under_addition}, $\Z_n$ is a monoid under addition modulo $n$.
		Now we must show that the set $\Z_n$ contains an inverse element for each of its members.
		Let $\bar{a} \in \Z_n$.
		Then since $(n-a) = (n-a) + kn$ for $k = 0 \in \Z$ implies that $(n-a) \equiv (n-a) \pmod n$, we know that $\overline{a - n} \in Z_n$ by the definition of congruence classes.
		Adding both of these elements, we have 
		\begin{align*}
			\bar{a} + \overline{n-a} 
			&= \overline{a + n - a} \\
			&= \overline{a + (-a) + n} \\ 
			&= \bar{n} \\
			&= \overline{n + a - a} \\
			&= \overline{n + (-a) + a} \\
			&= \overline{n-a} + \bar{a}
		\end{align*}
		by the commutativity of addition on the integers.
		Hence, $\bar{a} + \overline{n-a} = \overline{n-a} + \bar{a} = \bar{a}$, and so $\Z_n$ is a group under this operation.
		
		To show that it is abelian, let $\bar{b} \in \Z_n$.
		Again by the commutativity of integer addition, we have 
		$$\bar{a}+\bar{b} = \overline{a + b} = \overline{b + a} = \bar{b} + \bar{a}.$$
		Since both of these elements were arbitrary, we conclude that this must be true for all elements in $\Z_n$.
		Therefore, $\Z_n$ forms an abelian group under addition modulo $n$.	
	\end{proof}
	
	\subsection{Ring}
	
	\begin{definition}[Ring]\label{definition:ring}
		Let $R$ be an abelian group.
		Then $R$ is a ring if it satisfies the following axioms:
		\begin{enumerate}
			\item $R$ forms a monoid under a second binary operation $\circ$ that distributes over the group operation, and 
			\item the additive identity $0 \in R$ satisfies $0 \circ a = 0$ for all $a \in R$.
		\end{enumerate}
	\end{definition}
	
	\begin{prop}\label{proposition:Zn_is_a_ring}
		Let $n > 0$ be any integer. Then $\Z_n$ forms a ring under addition modulo $n$ and multiplication modulo $n$.
	\end{prop}
	
	\begin{proof}
		Let $n > 0$ be any integer.
		Then, by \cref{proposition:Zn_is_an_abelian_group_under_addition}, $\Z_n$ forms an abelian group under addition modulo $n$.
		Moreover, $\Z_n$ also forms a monoid under multiplication modulo $n$ --- a second binary operation.
		Next we must show that this multiplication distributes over this addition.
		Let $\bar{a}, \bar{b}, \bar{c} \in R$.
		Since multiplication distributes over addition on the integers, we have	
		\begin{align*}
			\bar{a} \cdot (\bar{b}+\bar{c}) 
			&= \bar{a} \cdot \overline{b+c} \\
			&= \overline{a (b+c)} \\
			&= \overline{a \cdot b+ac} \\
			&= \overline{a \cdot b} + \overline{a \cdot c} \\
			&= \bar{a} \cdot \bar{b} + \bar{a} \cdot \bar{c}
		\end{align*}
		for left distribution, and 
		\begin{align*}
			(\bar{a}+\bar{b}) \cdot \bar{c} 
			&= \overline{a + b} \cdot \bar{c} \\
			&= \overline{(a + b)c} \\
			&= \overline{a \cdot c + b \cdot c} \\
			&= \overline{a \cdot c} + \overline{b \cdot c} \\
			&= \bar{a} \cdot \bar{c} + \bar{b} \cdot \bar{c}
		\end{align*}
		for right distribution.
		Thus, this multiplication distributes over addition modulo $n$.
		
		Lastly, since $\bar{0} \cdot \bar{a} = \overline{0 \cdot a} = \bar{0}$, we conclude that the product of the additive identity, $\bar{0} \in R$, and any element in $R$ is equal to $\bar{0}$. Therefore, $\Z_n$, together with this addition and this multiplication, forms a ring.	
	\end{proof}
	
%	We will use the usual convention that multiplication takes precedence over addition, thus for instance $\bar{a} \cdot \bar{b} + \bar{c}$ means $(\bar{a} \cdot \bar{b}) + \bar{c}$, and not $\bar{a} \cdot (\bar{b} + \bar{c})$.

	
	\subsection{Field}
		
	\begin{definition}[Field]\label{definition:field}
		Let $F$ be a ring under two commutative binary operations.
		If every nonzero element in $R$ has a multiplicative inverse, then we say $F$ is a field.
	\end{definition}

	\begin{lemma}\label{lemma:all_a_relatively prime_p}
		Let $p$ be prime. If $\bar{a} \in \Z_p$ such that $\bar{a} \neq \bar{0}$, then $\gcd(a,p) = 1$.
	\end{lemma}
	
	\begin{proof}
		Let $p$ be prime, and let $\bar{a} \in \Z_p$ such that $\bar{a} \neq \bar{0}$.
		Now suppose for the sake of contradiction that $p \mid a$.
		Then $a = lp$ for some $l \in \Z$.
		By \cref{definition:congruence_class},
		$$\bar{a} = \{ a + kp : k\in\Z \} = \{ lp + kp : k\in\Z \} = \{ 0 + (l+k)p : k\in\Z \} = \bar{0},$$
		and so $\bar{a} = \bar{0}$.
		But we know $\bar{a} \neq \bar{0}$ and because of this contradiction we must conclude $p \nmid a$.
		Therefore, by \cref{prop:gcd(a,p)=1}, $\gcd(a,p) = 1$.
	\end{proof}
	
	\begin{theorem}\label{theorem:Zp_is_a_field}
		Let $p$ be prime. Then $\Z_p$ forms a field under addition modulo $n$ and multiplication module $n$.
	\end{theorem}
	
	\begin{proof}
		Let $p$ be prime.
		Since primes are integers greater than 1 and we know, by \cref{proposition:Zn_is_a_ring}, $\Z_n$ is a ring for any integer $n > 1$, then $\Z_p$ is a ring.
		Furthermore, this addition modulo $n$ is commutative since the underlying set of a ring forms an abelian group under this operation.
		Now we must show that multiplication modulo $n$ is also commutative.
		Let $\bar{a}, \bar{b} \in \Z_p$.
		Then, by the commutativity of integer multiplication, $\bar{a} \cdot \bar{b} = \overline{a \cdot b} = \overline{b \cdot a} = \bar{b} \cdot \bar{a}$ as required.
		
		Lastly, we will confirm that every nonzero element in $\Z_p$ has a multiplicative inverse.
		So, suppose $\bar{a} \neq \bar{0}$.
		Then by \cref{Lemma. a in Zp has a unique inverse in Zp}, we have $\gcd(a,p)=1$.
		Moreover, this implies that $ax + py = 1$ according to \cref{proposition:smallest_possible_linear_combination}.
		It follows that $ax = 1 + (-y)p$; that is, $ax \equiv 1 \pmod p$ by \cref{proposition:equivalent_conditions_for_congruent}.
		Hence, $\overline{a \cdot x} = \bar{a} \cdot \bar{x} = \bar{1}$.
		Therefore, every nonzero element in $\Z_p$ has a multiplicative inverse, and so we conclude that $\Z_p$ is a field. 
	\end{proof}
	
	\begin{cor}[Zero product property]\label{corollary:zero_product_property}
		Let $p$ be prime. If $\bar{a} \cdot \bar{b} = \bar{0}$, then either $\bar{a} = \bar{0}$ or $\bar{b} = \bar{0}$ for all $\bar{a}, \bar{b} \in \Z_p$.
	\end{cor}

	\begin{proof}
		Let $p$ be prime, and let $\bar{a}, \bar{b} \in \Z_p$ such that $\bar{a} \cdot \bar{b} = \bar{0}$.
		If both $\bar{a} = \bar{0}$ and $\bar{b} = \bar{0}$, then $$\bar{a} \cdot \bar{b} = \bar{0} \cdot \bar{0} = \overline{0 \cdot 0} = \bar{0}.$$
		Now let's suppose that $\bar{a}, \bar{b}$ are both not zero.
		It follows from \cref{theorem:Zp_is_a_field} that every nonzero element in $\Z_p$ has a multiplicative inverse.
		So, we proceed with two cases.
		
		\textbf{Case 1}. If $\bar{a} \neq \bar{0}$, then 
		\begin{align*}
			\bar{a} \cdot \bar{b} &= \bar{0} \\
			\bar{a}^{-1} \cdot \bar{a} \cdot \bar{b} &= \bar{a}^{-1} \cdot \bar{0} \\
			\bar{1} \cdot \bar{b} &= \bar{0} \\
			\bar{b} &= \bar{0}.
		\end{align*}
		
		\textbf{Case 2}. If $\bar{b} \neq \bar{0}$, then 
		\begin{align*}
		\bar{a} \cdot \bar{b} &= \bar{0} \\
		\bar{a} \cdot \bar{b} \cdot \bar{b}^{-1} &= \bar{0} \cdot \bar{b}^{-1} \\
		\bar{a} \cdot \bar{1} &= \bar{0} \\
		\bar{a} &= \bar{0}. \\
		\end{align*}
		Therefore, if $\bar{a} \cdot \bar{b} = \bar{0}$, then either $\bar{a} = \bar{0}$ or $\bar{b} = \bar{0}$ for all $\bar{a}, \bar{b} \in \Z_p$.
				
	\end{proof}
	
	
	% ==========================================================================
	% A Probabilistic Test for Compositeness
	% ==========================================================================	
	
	\section{A Probabilistic Test for Compositeness}\label{section:tests}

	\subsection{Fermat's Little Theorem}
	
%	\begin{theorem}[Fermat's Little Theorem]
%		Let $p \in \N$ be prime, and let $\bar{a} \in \Z_p$ such that $\bar{a} \neq \bar{0}$. Then
%		$$\bar{a}^{p-1} = \bar{1}.$$
%	\end{theorem}

	\begin{lemma}\label{lemma:unique_inverses_in_Zp}
		Let $p$ be prime. Then for each $\bar{a} \in \Z_p$ with $\bar{a} \neq \bar{0}$, there exists a unique multiplicative inverse.
	\end{lemma}
	
	\begin{proof}
		Let $p$ be prime, and let $\bar{a} \in \Z_p, \bar{a} \neq \bar{0}$. 
		Since $\Z_p$ is a field, we know $\bar{a}$ has a multiplicative inverse in $\Z_p$.
		Suppose both $\bar{x}, \bar{y} \in \Z_p$ are multiplicative inverses of $\bar{a}$.
		Then $\bar{a} \cdot \bar{x} = \bar{1}$ and $\bar{a} \cdot \bar{y} = \bar{1}$.
		Thus, $\bar{a} \cdot \bar{x} = \bar{a} \cdot \bar{y}$, and by left cancellation we have $\bar{x}=\bar{y}$.
		Therefore, for each $\bar{a} \in \Z_p$ with $\bar{a} \neq \bar{0}$, there exists a unique multiplicative inverse.		
	\end{proof}

	\begin{table}[h!]
		\centering
		\begin{minipage}{0.48\textwidth}
			\centering
			\caption{Multiplication in $\Z_7$}
			\label{Multiplication in Z_7}
			\begin{tabular}{|c|c|c|c|c|c|c|c|}
				\hline
			$\cdot$		& $\bar{0}$ & $\bar{1}$ & $\bar{2}$ & $\bar{3}$ & $\bar{4}$ & $\bar{5}$ & $\bar{6}$ \\ \hline
			$\bar{0}$	& $\bar{0}$ & $\bar{0}$ & $\bar{0}$ & $\bar{0}$ & $\bar{0}$ & $\bar{0}$ & $\bar{0}$ \\ \hline
			$\bar{1}$	& $\bar{0}$ & $\bar{1}$ & $\bar{2}$ & $\bar{3}$ & $\bar{4}$ & $\bar{5}$ & $\bar{6}$ \\ \hline
			$\bar{2}$	& $\bar{0}$ & $\bar{2}$ & $\bar{4}$ & $\bar{6}$ & $\bar{1}$ & $\bar{3}$ & $\bar{5}$ \\ \hline
			$\bar{3}$	& $\bar{0}$ & $\bar{3}$ & $\bar{6}$ & $\bar{2}$ & $\bar{5}$ & $\bar{1}$ & $\bar{4}$ \\ \hline
			$\bar{4}$	& $\bar{0}$ & $\bar{4}$ & $\bar{1}$ & $\bar{5}$ & $\bar{2}$ & $\bar{6}$ & $\bar{3}$ \\ \hline
			$\bar{5}$	& $\bar{0}$ & $\bar{5}$ & $\bar{3}$ & $\bar{1}$ & $\bar{6}$ & $\bar{4}$ & $\bar{2}$ \\ \hline
			$\bar{6}$	& $\bar{0}$ & $\bar{6}$ & $\bar{5}$ & $\bar{4}$ & $\bar{3}$ & $\bar{2}$ & $\bar{1}$ \\ \hline
			\end{tabular}
		\end{minipage}
		\hfill
		\begin{minipage}{0.48\textwidth}
			\centering
			\caption{Multiplication in $\Z_8$}
			\label{Multiplication in Z_8}
			\begin{tabular}{|c|c|c|c|c|c|c|c|c|}
				\hline
			$\cdot$		& $\bar{0}$ & $\bar{1}$ & $\bar{2}$ & $\bar{3}$ & $\bar{4}$ & $\bar{5}$ & $\bar{6}$ & $\bar{7}$ \\ \hline
			$\bar{0}$	& $\bar{0}$ & $\bar{0}$ & $\bar{0}$ & $\bar{0}$ & $\bar{0}$ & $\bar{0}$ & $\bar{0}$ & $\bar{0}$ \\ \hline
			$\bar{1}$	& $\bar{0}$ & $\bar{1}$ & $\bar{2}$ & $\bar{3}$ & $\bar{4}$ & $\bar{5}$ & $\bar{6}$ & $\bar{7}$ \\ \hline
			$\bar{2}$	& $\bar{0}$ & $\bar{2}$ & $\bar{4}$ & $\bar{6}$ & $\bar{0}$ & $\bar{2}$ & $\bar{4}$ & $\bar{6}$ \\ \hline
			$\bar{3}$	& $\bar{0}$ & $\bar{3}$ & $\bar{6}$ & $\bar{1}$ & $\bar{4}$ & $\bar{7}$ & $\bar{2}$ & $\bar{5}$ \\ \hline
			$\bar{4}$	& $\bar{0}$ & $\bar{4}$ & $\bar{0}$ & $\bar{4}$ & $\bar{0}$ & $\bar{4}$ & $\bar{0}$ & $\bar{4}$ \\ \hline
			$\bar{5}$	& $\bar{0}$ & $\bar{5}$ & $\bar{2}$ & $\bar{7}$ & $\bar{4}$ & $\bar{1}$ & $\bar{6}$ & $\bar{3}$ \\ \hline
			$\bar{6}$	& $\bar{0}$ & $\bar{6}$ & $\bar{4}$ & $\bar{2}$ & $\bar{0}$ & $\bar{6}$ & $\bar{4}$ & $\bar{2}$ \\ \hline
			$\bar{7}$	& $\bar{0}$ & $\bar{7}$ & $\bar{6}$ & $\bar{5}$ & $\bar{4}$ & $\bar{3}$ & $\bar{2}$ & $\bar{1}$ \\ \hline
			\end{tabular}
		\end{minipage}
	\end{table}	
	
	\begin{lemma}\label{lemma:1_and_p-1_own_inverses}
		Let $p$ be prime. If $\bar{a} \in \Z_p$ is its own multiplicative inverse, then $\bar{a} = \bar{1}$ or $\bar{a} = \overline{p-1}$.
	\end{lemma}
	
	\begin{proof}
		Let $p \in \N$ be prime, and let $\bar{a} \in \Z_p$ be its own multiplicative inverse.
		Then $\bar{a} \cdot \bar{a} = \bar{a}^2 = \bar{1}$; that is, $\bar{a}^2 - \bar{1} = (\bar{a} + \bar{1})(\bar{a} - \bar{1}) = \bar{0}$. 
		By \cref{corollary:zero_product_property}, since $\bar{a} + \bar{1} \in \Z_p$ and $\bar{a} - \bar{1} \in \Z_p$ and $(\bar{a} + \bar{1})(\bar{a} - \bar{1}) = \bar{0}$, then either $(\bar{a} + \bar{1}) = \bar{0}$ or $(\bar{a} - \bar{1}) = \bar{0}$. 
		We will consider both cases.
		
		\textbf{Case 1}. If $(\bar{a} + \bar{1}) = \bar{0}$, then $\bar{a} = -\bar{1} = \overline{p-1}$.
		
		\textbf{Case 2}. If $(\bar{a} - \bar{1}) = \bar{0}$, then $\bar{a} = \bar{1}$.
		
		\noindent Therefore, $\bar{a} = \bar{1}$ or $\bar{a} = \overline{p-1}$.
	\end{proof}
	
	
	\begin{theorem}[Fermat's Little Theorem \cite{pommersheim}]\label{theorem:fermats_little_theorem}
		Let $p$ be prime, and let $\bar{a} \in \Z_p, \bar{a} \neq \bar{0}$. Then $$ \bar{a}^{p-1} = \bar{1}.$$
	\end{theorem}
	
	\begin{proof}
		Let $p$ be prime, and let $\bar{a} \in \Z_p, \bar{a} \neq \bar{0}$.
		By \cref{lemma:unique_inverses_in_Zp}, we know that $\Z_p$ contains a unique inverse for each of its elements. 
		Furthermore, $\bar{1}^{-1} = \bar{1}$ and $\overline{p-1}^{-1} = \overline{p-1}$ by \cref{lemma:1_and_p-1_own_inverses}.
		Thus, $\bar{1} \cdot \bar{2} \cdot \bar{3} \cdots \overline{p-1} = \bar{1} \cdot \overline{p-1} = \overline{p-1}$. 
		Then 
		\begin{align*}
			(\bar{a} \cdot \bar{1})(\bar{a} \cdot \bar{2}) \cdots (\bar{a} \cdot \overline{p-1})
			&= \underbrace{\bar{a} \cdot \bar{a} \cdots \bar{a}}_{p-1 \text{ times}} \cdot \bar{1} \cdot \bar{2} \cdots \bar{a} \cdots \bar{a}^{-1} \cdots \overline{p-1} \\
			&= \bar{a}^{p-1} \cdot \overline{p-1}.
		\end{align*}
		Moreover, since this multiplication is a binary operation, we know that each product is equal to a unique element in $\Z_p$.
		Thus, $(\bar{a} \cdot \bar{1})(\bar{a} \cdot \bar{2}) \cdots (\bar{a} \cdot \overline{p-1}) = \bar{1} \cdot \bar{2} \cdots \overline{p-1}$, where the right-hand side is some permutation of the elements in $\Z_p$.
		Hence,
		\begin{align*}
			\bar{a}^{p-1} \cdot \overline{p-1} 
			&= \bar{1} \cdot \bar{2} \cdots \overline{p-1} \\
			\bar{a}^{p-1} \cdot \overline{p-1}
			&= \overline{p-1} \\
			\bar{a}^{p-1}\cdot \overline{p-1} \cdot \overline{p-1}
			&= \overline{p-1}	\cdot \overline{p-1} \\
			\bar{a}^{p-1}\cdot \bar{1} 
			&= \bar{1}  \\
			\bar{a}^{p-1}
			&= \bar{1}.
		\end{align*}
		Therefore, if $p$ is prime, then $\bar{a}^{p-1} = \bar{1}$ for all $\bar{a} \in \Z_p, \bar{a} \neq \bar{0}$.
	\end{proof}
	
%	\begin{cor}[Fermat Test for Compositeness]
%		Suppose $n > 0$ be any integer. If there exists $\bar{a} \in \Z_n, \bar{a} \neq \bar{0}$, such that $$\bar{a}^{n-1} \neq \bar{1},$$ then $n$ is composite. In this case, the integer $a$ is called a Fermat witness to the compositeness of $n$.
%	\end{cor}
%	
%	\begin{definition}[Carmichael number]
%		A composite number $n$ is called a \textbf{Carmichael number} if for every integer $a$, $$\gcd(a, n) = 1 \implies a^{n - 1} \equiv 1 \pmod{n}.$$
%	\end{definition}
%	
%	The first Carmichael number is $561 = 3 \cdot 11 \cdot 17$. \cite{oeis:carmichael}  They are quite sparse, with only 43 Carmichael numbers in the first one million natural numbers. In 1994, mathematicians Alford, Granville, and Pomerance proved that there are infinitely many of these numbers. \cite{pommersheim}
	
	\subsection{Miller-Rabin Test}

	\begin{example}\label{example:miller-rabin_prime_modulus}
		Since $29$ is prime, we know that $\bar{a}^{28} = \bar{1}$ for all $\bar{a} \in \Z_{29}$ by \cref{theorem:fermats_little_theorem}. 
		In other words, $\bar{a}^{28}-\bar{1}=\bar{0}$.
		Since $\Z_{29}$ is a field, this polynomial can be factored using the difference of squares.
		Then
		\begin{align*}
		\bar{a}^{28}-\bar{1}
		&= (\bar{a}^{14}+\bar{1})(\bar{a}^{14}-\bar{1}) \\
		&= (\bar{a}^{14}+\bar{1})(\bar{a}^{7}+\bar{1})(\bar{a}^{7}-\bar{1}) \\
		&= \bar{0}.
		\end{align*}
		Since 7 is odd, it is not possible to use the difference of squares to factor $(\bar{a}^{7}-\bar{1})$ any further.
		By \cref{corollary:zero_product_property}, we know $(\bar{a}^{14}+\bar{1})(\bar{a}^{7}+\bar{1})(\bar{a}^{7}-\bar{1}) = \bar{0}$ implies that either $(\bar{a}^{14}+\bar{1})=\bar{0}$ or $(\bar{a}^{7}+\bar{1})=\bar{0}$ or $(\bar{a}^{7}-\bar{1})=\bar{0}$.
		Even if we randomly select an $\bar{a} \in \Z_{29}$, we expect this to still be true.
		So, let $\bar{a} = \bar{7}$.
		Then we have
		\begin{align*}
		(\bar{7}^{14}+\bar{1})&=\bar{2} \\
		(\bar{7}^{7}+\bar{1})&=\bar{2} \\
		(\bar{7}^{7}-\bar{1})&=\bar{0},
		\end{align*}
		as we expected.	
	\end{example}

	\begin{algorithm}[Miller-Rabin Test for Compositeness]\label{algorithm:miller-rabin}
		Let $n \geq 3$ be any odd integer. 
		Then there exists an integer $k > 0$ such that $2^k$ is the largest power of two that divides $n-1$.
		If there exists $\bar{a} \in \Z_n$ such that  
		$$\bar{a}^{\frac{n-1}{2^k}} \neq \bar{1}$$ and 
		$$\bar{a}^{\frac{n-1}{2^h}} \neq -\bar{1},$$ 
		for all $h \in \Z : 1 \leq h \leq k$, then $n$ is composite. 
		In this case, the integer $a$ is called a Miller-Rabin witness to the compositeness of $n$.
	\end{algorithm}
	
	\begin{example}
		We would like to use \cref{algorithm:miller-rabin} to test the compositeness of 169. Since $2^3$ is the largest power of two that divides 168, we must find an $\bar{a}\in \Z_{169}$ such that $\bar{a}^{\frac{168}{2^3}} \neq \bar{1}$ and $\bar{a}^{\frac{168}{2^h}} \neq -\bar{1}$ for all $h$, $h=1,2,3$. So, we randomly choose $\overline{19} \in \Z_{169}$ and find that
		\begin{align*}
			\overline{19}^{\frac{168}{2^3}} &= \overline{70} \\
			\overline{19}^{\frac{168}{2^2}} &= -\bar{1} \\
			\overline{19}^{\frac{168}{2^1}} &= \bar{1}. \\
		\end{align*}
		Because $\overline{19}^{\frac{168}{2^2}} = -\bar{1}$, we cannot conclude that $169$ is composite. So we randomly select a different $\bar{a} \in \Z_{169}$, namely $\bar{a} = \overline{145}$, and this time discover that
		\begin{align*}
		\overline{145}^{\frac{168}{2^3}} &= \overline{18} \\
		\overline{145}^{\frac{168}{2^2}} &= \overline{155} \\
		\overline{145}^{\frac{168}{2^1}} &= \overline{27}. \\
		\end{align*}
		Hance, $145$ is a Miller-Rabin witness to the compositeness of $169$ and we conclude that $169$ is not prime. 
	\end{example}
	
	
	\subsection{Effectiveness of the Miller-Rabin Test}	
	
	
	%Miller-Rabin Probability Test.
	%Fermat Test for Compositeness.
	%Characterization of Carmichael Numbers.
	
	%\subsection{Lukas-Lemer Test}
	
%	\subsection{Solovay-Strassen Test}
%
%%	\subsection{Quadratic Congruences}
%	
%	\subsubsection{Quadratic Residues}
%	
%	\begin{definition}
%		Let $n \in \N$ and let $\bar{a} \in \Z_n$ with $\bar{a} \neq \bar{0}$. We say $\bar{a}$ is a \textbf{quadratic residue} modulo $n$ if there exists $\bar{b} \in \Z_n$ such that $\bar{b}^2 = \bar{a}$. If there is no such $\bar{b}$, then we say that $\bar{a}$ is a quadratic \textbf{nonresidue} modulo $n$. \cite{pommersheim}
%	\end{definition}
%	
%	\begin{example}
%		
%		Let $n = 7$. In \cref{Z_7-exponent-table}, each $\bar{a} \in \Z_7$ has its own column. The second row shows $\bar{a}^2$. 
%		
%		\begin{table}[h]
%			\centering
%			\caption{$\Z_7$ Exponent Table}
%			\label{Z_7-exponent-table}
%			\def\arraystretch{1.5}
%			\begin{tabular}{|l|l|l|l|l|l|l|l|}
%				\hline
%				$\bar{a}$ 	& $\bar{0}$ & $\bar{1}$ & $\bar{2}$ & $\bar{3}$ & $\bar{4}$ & $\bar{5}$ & $\bar{6}$ \\ \hline
%				$\bar{a}^2$ & $\bar{0}$ & $\bar{1}$ & $\bar{4}$ & $\bar{2}$ & $\bar{2}$ & $\bar{4}$ & $\bar{1}$ \\ \hline
%				$\bar{a}^3$ & $\bar{0}$ & $\bar{1}$ & $\bar{1}$ & $\bar{6}$ & $\bar{1}$ & $\bar{6}$ & $\bar{6}$ \\ \hline
%				$\bar{a}^4$ & $\bar{0}$ & $\bar{1}$ & $\bar{2}$ & $\bar{4}$ & $\bar{4}$ & $\bar{2}$ & $\bar{1}$ \\ \hline
%				$\bar{a}^5$ & $\bar{0}$ & $\bar{1}$ & $\bar{4}$ & $\bar{5}$ & $\bar{2}$ & $\bar{3}$ & $\bar{6}$ \\ \hline
%				$\bar{a}^6$ & $\bar{0}$ & $\bar{1}$ & $\bar{1}$ & $\bar{1}$ & $\bar{1}$ & $\bar{1}$ & $\bar{1}$ \\ \hline
%				$\bar{a}^7$ & $\bar{0}$ & $\bar{1}$ & $\bar{2}$ & $\bar{3}$ & $\bar{4}$ & $\bar{5}$ & $\bar{6}$ \\ \hline
%				% $\bar{x}^2$ & $\bar{0}$ & $\bar{1}$ & $\bar{1}$ & $\bar{6}$ & $\bar{1}$ & $\bar{6}$ & $\bar{6}$ \\ \hline
%			\end{tabular}
%		\end{table}
%		
%		We notice that there are two elements $\bar{b} \in \Z_7$, namely $\bar{1}$ and $\bar{6}$, such that $\bar{b}^2 = \bar{1} = \bar{a}$. Thus, $\bar{1}$ is a quadratic residue modulo $7$. Clearly, $\bar{2}$ and $\bar{4}$ are also quadratic residues here. On the other hand, there is no element in $\Z_7$ that equals $\bar{3}$, $\bar{5}$, or $\bar{6}$ when squared. So, we say that $\bar{3}$, $\bar{5}$, or $\bar{6}$ are quadratic nonresidues modulo $7$.
%		
%	\end{example}
%	
%	\subsubsection{The Legendre Symbol}
%	
%	\begin{definition}
%		Let $p > 2$ be prime, and let $a \in \Z$. The \textbf{Legendre symbol}, $\left( \frac{a}{p} \right)$, is defined by $$ \left( \frac{a}{p} \right) = \begin{cases}
%		1 &\text{if } \bar{a} \text{ is a quadratic residue modulo } p \\
%		-1 &\text{if } \bar{a} \text{ is a quadratic nonresidue modulo } p \\
%		0 &\text{if } \bar{a} = \bar{0} \text{ in } \Z_p (\text{i.e., if } p \mid a).
%		\end{cases}$$\cite{pommersheim}
%	\end{definition}	
%	
%	\subsubsection{Jacobi Symbol}
%	
%	\begin{theorem}[Fundamental Theorem of Arithmetic]\label{theorem:fta}
%		Let $n \in \N : n > 1$. Then $n$ may be written as a product of one or more prime numbers. Furthermore, the prime factorization of $n$ is unique up to order. \cite{pommersheim}
%	\end{theorem}
%	
%	\begin{definition}
%		Let $n$ be an odd integer with prime factorization $n = p^{k_1}_1 \cdot p^{k_2}_2 \cdot \cdots \cdot p^{k_m}_m$, and let $a$ be any integer relatively prime to $n$. Then we define the $\textbf{Jacobi symbol} \left( \frac{a}{n} \right)$ as follows: $$ \left( \frac{a}{n} \right) = \left( \frac{a}{p_1} \right)^{k_1} \cdot \left( \frac{a}{p_2} \right)^{k_2} \cdot \cdots \cdot \left( \frac{a}{p_m} \right)^{k_m}, $$ where the expression on the right side is a product of powers of Legendre symbols.\cite{pommersheim}
%	\end{definition}
%	
%	
%	
%	
%	\begin{definition}[Euler-Jacobi pseudoprime]
%		An \textbf{Euler-Jacobi pseudoprime} to a base $a$ is an odd composite number $n$ such that $(a,n) = 1$ and the Jacobi symbol $\left( \frac{a}{n} \right)$ satisfies $$\left( \frac{a}{n} \right) \equiv a^{\frac{n-1}{2}} \pmod{n}. $$
%	\end{definition}
%	
%	The first few base-2 Euler-Jacobi pseudoprimes are $561, 1105, 1729, 1905, 2047, 2465,\ldots$. \cite{oeis:pseudoprimes_base2}
%	
	

	% ==========================================================================
	% Suppose Vector Machine
	% ==========================================================================	
		
	\section{Support Vector Machine for Binary Classification}\label{svm}
	
	\begin{definition}[Support Vector Machine]
		A Support Vector Machine (SVM) is a learning system that uses a hypothesis space of linear functions in a high dimensional feature space, trained with a learning algorithm from optimisation theory that implements a learning bias derived from statistical learning theory.\cite{christianini}
	\end{definition}
	
	
	\begin{itemize}
		\item target function --- underlying function that maps inputs to outputs (if it exists)
		\item solution --- estimate of the target function by learning algorithm (also called the decision function in classification algorithms)
		\item hypothesis space --- a set or class of candidate solutions (known as hypotheses)
		\item learning algorithm --- uses training data to select a hypothesis 
		\item features --- the quantities used to describe the data
		\item attributes --- original quantities from data
	\end{itemize}
	
	``SVM decision function depends on some subset of the training data, called the support vectors."
		
%	\subsection{Vector Spaces}
%
%	\begin{definition}[Hyperplane]
%		Let $a_1,a_2,\ldots,a_n$ be scalars not all equal to 0. Then the set $S$ consisting of all vectors $$x = \begin{pmatrix} x_1 \\ x_2 \\ \vdots \\ x_n \end{pmatrix} \in \R^n$$ such that $a_1x_1 + a_2x_2 + \cdots + a_nx_n = c$ for a constant $c$ is a subspace of $\R^n$ called a \textbf{hyperplane}.\cite{mathworld:hyperplane}
%	\end{definition}
%	
%	\subsection{Inner Product Spaces}
%	
%	\begin{definition}[Gram matrix]
%		Given a set $V$ of $m$ vectors (points in $\R^m$), the \textbf{Gram matrix} $G$ is the matrix of all possible inner products of $V$, i.e., $$ g_{ij}  = v_i^Tb_j, $$ where $A^T$ denotes the transpose. The Gram matrix determines the vectors $v_i$ up to isometry. \cite{mathworld:gram_matrix}
%	\end{definition}
%	
%	\subsubsection{Hilbert Spaces}
%	
%	\begin{definition}[Hilbert Space]
%		A \textbf{Hilbert space} is a vector space $H$ with an inner product $\langle f,g \rangle$ such that the norm defined by $$ |f| = \sqrt{\langle f,f \rangle} $$ turns $H$ into a complete metric space. If the metric defined by the norm is not complete, then H is instead known as an inner product space. \cite{mathworld:hilbert_space}
%	\end{definition}
%	
%	\subsection{Operators, Eigenvalues and Eigenvectors}
%	
	\subsection{Kernel Induced Feature Spaces} % christianini chapters 2 and 3
	
	``Project the data into a higher dimensional feature space to increase the computational power of the linear learning machines." \cite{christianini}
	
	\begin{definition}[Kernel \cite{christianini}]\label{definition:kernel}
		A kernel is a function $K$, such that for all $x, z \in X$
		$$ K(x,z) = (\phi(x) \cdot \phi(z)),$$
		where $\phi$ is a mapping from $X$ to an (inner product) feature space $F$. 
	\end{definition}
	
	\begin{example}
		Map input space into new space.
	\end{example}
	
	``not scale invariant, so it is highly recommended to scale your data. For example, scale each attribute on the input vector $X$ to [0,1] or [-1,+1], or standardize it to have mean 0 and variance 1."
	
	
	
	
	
	\subsection{Learning Bias} % christianini chapter 4
	
	\subsection{Learning Algorithm} % christianini chapters 5 and 7
	
	%\subsection{Cononical Hyperplane}
	
	%\subsection{Lagrangian Optimization}
	
	% ==========================================================================
	% Method
	% ==========================================================================	
		
	\section{Method}\label{method}
		
	\subsection{Features}
	
	\subsubsection{Training Data}
	
	We will use $X \subset \R^n$ and $Y = (-1, 1)$ for binary classification. 
	
	\begin{definition}[Training Set]
		A \textbf{training set} is a collection of training examples, which are also called training data. It is usually denoted by $S = \{ (x_1, y_1),(x_2, y_2),\ldots, (x_n, y_n)\} \subset X \times Y$, where $n$ is the number of examples. We refer to $x_i$ as examples or instances and $y_i$ as their labels.\cite{christianini}
	\end{definition}
	
	
	\subsubsection{Base-b Representations}	
	
	\begin{theorem}[Well Ordering Principle]\label{theorem:well_ordering_principle}
		Every nonempty set of positive integers contains a smallest member. \cite{mathworld:well_ordering_principle}
	\end{theorem}
	
	\begin{proof}
		Let $A \subseteq \Z^+ : A \neq \emptyset$. Assume for the sake of contradiction that $A$ does not have a smallest member. Then $1 \not\in A$ since $1 \leq n$ for all $n \in \Z^+$. Now we assume positive integers $1, 2, \ldots, k \not\in A$. Then $k + 1 \not\in A$ since $k + 1$ would be the smallest member of $A$. Thus, by strong induction $A = \emptyset$, a contradiction. Therefore, $A$ must contain a smallest member.
	\end{proof}
	
	\begin{theorem}[The Division Algorithm]\label{theorem:division_algorithm}
		Let $a, b \in \Z : b > 0$. Then there exist unique $q, r \in \Z$ such that $ a = bq + r $ and $0 \leq r < b$. \cite{pommersheim}
	\end{theorem}	
	
	\begin{proof}
		Let $a, b \in \Z : b > 0$, and let $R = \{ x \in \Z : a - xb \geq 0 \}$. First, we notice that $R \neq \emptyset$ for if $a \leq 0$ and $x = a$ then $a - ab = a(1-b) \geq 0$, and if $a > 0$ and $x = 0$ then $a - 0 \cdot b \geq 0$. 
		
		Thus, by \cref{theorem:well_ordering_principle}, $R$ contains a smallest member, which we will call $r$. Hence, $\exists q \in \Z : r = a - qb \geq 0$. Now suppose for the sake of contradiction that $r \geq b$. 
		Then $r = b + n \geq 0$ for some $n \in \Z: n \geq 0$. Hence, $r = b + n = a - qb$ or $n = a - qb - b = a - (q + 1)b \in R$. But $n = a - (q + 1)b < a - ab = r$ is impossible since $r$ is the smallest member of $R$. So, it must be that $r < b$. Therefore, we have found $q, r \in \Z$ such that $ a = bq + r $ and $0 \leq r < b$.
		
		To complete the proof, we must show that $q$ and $r$ are unique. So, we let $q^\prime, r^\prime \in \Z$ and assume $a = q^\prime b + r^\prime$ such that $0 \leq r^\prime < b$. Thus, $a = qb - r = q^\prime b + r^\prime$. In other words,  $r - r^\prime = q^\prime b - qb $ or $r - r^\prime = b (q^\prime - q)$. Without loss of generality, we may assume that $r^\prime \leq r$ such that $0 \leq r - r^\prime \leq r < b$. Therefore, $r - r^\prime$ must be a nonnegative multiple of $b$, such as $0, b, 2b, 3b, \ldots$, but $r - r^\prime < b$ implies that $r - r^\prime = 0$ or $r = r^\prime$. Also, since $b > 0$, $r - r^\prime = 0 = b (q^\prime - q) = b \cdot 0$. Thus, $q^\prime - q = 0$ or $q^\prime = q$. Therefore, $q$ and $r$ are unique.
		
		%Furthermore, we know $0 \leq q^\prime - q$ because $0 \leq r - r^\prime$ and $0 < b$.
		
		%By \cref{definition:divisibility}, $r - r^\prime = b (q^\prime - q) $ if and only if $b | r - r^\prime$. This implies $b \leq r - r^\prime$ according to \cref{lemma:divisibility}. 
		% Since $r - r^\prime < 0$ or $r^\prime > r$ contradicts the assumption that $r^\prime \leq r$, we know $r - r^\prime = 0$ or $r^\prime = r$. It follows that $r - r^\prime = 0 = b (q^\prime - q)$ if and only if $q^\prime - q = 0$ or $q^\prime = q$, since $b > 0$. Therefore, $q$ and $r$ are unique. 
	\end{proof}
	
	
	%\subsection{Base-b Representations}
	
	\begin{cor}
		Let $b \in \Z : b \geq 2$. Then every $N \in \Z : N > 0$ can be expressed uniquely in the form $N = a_kb^k + a_{k-1}b^{b-1} + \cdots + a_1 b + a_0$, where $a_0, a_1, \ldots, a_k$ are nonnegative integers less than $b$, $a_k \neq 0$, and $k \geq 0$. \cite{koshy}
	\end{cor}
	
	%\subsection{Operations in Nondecimals Bases}
	
	
	
	\subsection{Training}
		
		\subsubsection{Finding the Best Parameters}
		
		\begin{itemize}
			\item C: penalty parameter C of the error term
			\item kernel: 
			\begin{itemize}
				\item linear
				\item polynomial
				\item rbf
				\item sigmoid
			\end{itemize}
			\item degree: int, degree of polynomial kernel function for 'poly'
			\item gamma: kernel coefficients for 'rbf', 'poly' and 'sigmoid'
			\item coef0: independent term in kernel function for 'poly' and 'sigmoid'
			\item probability: whether to enable probability estimates
			\item shrinking: whether to use the shrinking heuristic
			\item tol: tolerance for stopping criterion
			\item cache\_size: specify the size of the kernel cache in MB
			\item class\_weight: if not given, all classes are suppose to have weight one
			\item verbose: enable verbose output
			\item max\_iter: hard limit on iterations within solver, or -1 for no limit
			\item decision\_function\_shape: for more than 2 classes
			\item random\_state: the seed of the pseudo random generator to use when shuffling the data for probability estimation
		\end{itemize}
	
	\subsection{Testing}
	

	% ==========================================================================
	% Conclusion
	% ==========================================================================	
		
	\section{Conclusion}\label{conclusions}
	
	\clearpage
	
	
	\begin{appendix}
	\section{Implementation of Fermat's Test}
	
	\begin{lstlisting}[frame=single]
	def fermats_test(n):
		nonwitnesses = []
		witnesses = []
		for a in range(1,n):
			right_hand_side = pow(a,n-1,n)
			if right_hand_side is 1:
				nonwitnesses.append(a)
			else:
				witnesses.append(a)
		return [nonwitnesses, witnesses]
	\end{lstlisting}
	
	Between 1 and 560 (inclusive), we have 320 integers $a$ such that for $\bar{a} \in \Z_{561}$, we have $\bar{a}^{650} = \bar{1}$.
	
	\begin{seqsplit}
	1, 2, 4, 5, 7, 8, 10, 13, 14, 16, 19, 20, 23, 25, 26, 28, 29, 31, 32, 35, 37, 38, 40, 41, 43, 46, 47, 49, 50, 52, 53, 56, 58, 59, 61, 62, 64, 65, 67, 70, 71, 73, 74, 76, 79, 80, 82, 83, 86, 89, 91, 92, 94, 95, 97, 98, 100, 101, 103, 104, 106, 107, 109, 112, 113, 115, 116, 118, 122, 124, 125, 127, 128, 130, 131, 133, 134, 137, 139, 140, 142, 145, 146, 148, 149, 151, 152, 155, 157, 158, 160, 161, 163, 164, 166, 167, 169, 172, 173, 175, 178, 179, 181, 182, 184, 185, 188, 190, 191, 193, 194, 196, 197, 199, 200, 202, 203, 205, 206, 208, 211, 212, 214, 215, 217, 218, 223, 224, 226, 227, 229, 230, 232, 233, 235, 236, 239, 241, 244, 245, 247, 248, 250, 251, 254, 256, 257, 259, 260, 262, 263, 265, 266, 268, 269, 271, 274, 277, 278, 280, 281, 283, 284, 287, 290, 292, 293, 295, 296, 298, 299, 301, 302, 304, 305, 307, 310, 311, 313, 314, 316, 317, 320, 322, 325, 326, 328, 329, 331, 332, 334, 335, 337, 338, 343, 344, 346, 347, 349, 350, 353, 355, 356, 358, 359, 361, 362, 364, 365, 367, 368, 370, 371, 373, 376, 377, 379, 380, 382, 383, 386, 388, 389, 392, 394, 395, 397, 398, 400, 401, 403, 404, 406, 409, 410, 412, 413, 415, 416, 419, 421, 422, 424, 427, 428, 430, 431, 433, 434, 436, 437, 439, 443, 445, 446, 448, 449, 452, 454, 455, 457, 458, 460, 461, 463, 464, 466, 467, 469, 470, 472, 475, 478, 479, 481, 482, 485, 487, 488, 490, 491, 494, 496, 497, 499, 500, 502, 503, 505, 508, 509, 511, 512, 514, 515, 518, 520, 521, 523, 524, 526, 529, 530, 532, 533, 535, 536, 538, 541, 542, 545, 547, 548, 551, 553, 554, 556, 557, 559, 560
	\end{seqsplit}
		
	There are 240 Fermat's witnesses to the compositeness of 561, namely all the multiples of 3, 11, 17 that are less than or equal to 560. 
	
	\begin{seqsplit}
	3, 6, 9, 11, 12, 15, 17, 18, 21, 22, 24, 27, 30, 33, 34, 36, 39, 42, 44, 45, 48, 51, 54, 55, 57, 60, 63, 66, 68, 69, 72, 75, 77, 78, 81, 84, 85, 87, 88, 90, 93, 96, 99, 102, 105, 108, 110, 111, 114, 117, 119, 120, 121, 123, 126, 129, 132, 135, 136, 138, 141, 143, 144, 147, 150, 153, 154, 156, 159, 162, 165, 168, 170, 171, 174, 176, 177, 180, 183, 186, 187, 189, 192, 195, 198, 201, 204, 207, 209, 210, 213, 216, 219, 220, 221, 222, 225, 228, 231, 234, 237, 238, 240, 242, 243, 246, 249, 252, 253, 255, 258, 261, 264, 267, 270, 272, 273, 275, 276, 279, 282, 285, 286, 288, 289, 291, 294, 297, 300, 303, 306, 308, 309, 312, 315, 318, 319, 321, 323, 324, 327, 330, 333, 336, 339, 340, 341, 342, 345, 348, 351, 352, 354, 357, 360, 363, 366, 369, 372, 374, 375, 378, 381, 384, 385, 387, 390, 391, 393, 396, 399, 402, 405, 407, 408, 411, 414, 417, 418, 420, 423, 425, 426, 429, 432, 435, 438, 440, 441, 442, 444, 447, 450, 451, 453, 456, 459, 462, 465, 468, 471, 473, 474, 476, 477, 480, 483, 484, 486, 489, 492, 493, 495, 498, 501, 504, 506, 507, 510, 513, 516, 517, 519, 522, 525, 527, 528, 531, 534, 537, 539, 540, 543, 544, 546, 549, 550, 552, 555, 558
	\end{seqsplit}
	
	\listoffigures
	
	\listoftables
	
	\end{appendix}
		
		
	\newpage
		
	\bibliographystyle{plain}
	\bibliography{refs.bib}
	
	

\end{document}